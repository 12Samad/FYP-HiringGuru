\chapter{Introduction} \label{sec
}

\section{Purpose of the Investigation}

The purpose of this investigation is to design and develop an AI-powered application that enables users to practice and enhance their interview skills. By creating a simulated interview environment, this application aims to assess key aspects of a candidate’s performance—such as body language, facial expressions, and verbal responses—through real-time analysis. The tool seeks to provide job seekers with meaningful feedback, helping them to better prepare for actual interview situations and improve their overall employability.

\section{Problem Being Investigated}

Despite the availability of abundant career guidance resources, candidates often struggle to gain insights into the quality of their interview skills. Traditional interview preparation methods lack interactive feedback on posture, facial expressions, and response delivery, which are critical components of a successful interview. The absence of such personalized and interactive feedback can hinder candidates’ ability to refine their presentation, leading to suboptimal interview performances. This investigation addresses the need for an advanced tool that uses AI to close this gap in interview preparation.

\section{Background and Importance of the Problem}

Interview preparation has evolved with advancements in technology, but many existing solutions are limited to static resources or basic question-and-answer formats. Previous research has demonstrated the effectiveness of feedback in improving interview skills, but few tools provide real-time analysis. By leveraging AI, facial recognition, and body language detection, this project builds on previous efforts to offer a more dynamic and personalized experience. Studies have shown that such technologies can significantly impact confidence and presentation in job interviews, underlining the importance of this application \cite{test}.

\section{Thesis and General Approach}

This thesis proposes that an AI-driven, interactive interview practice platform can improve job candidates' readiness by providing actionable feedback on their performance. The approach involves using computer vision and natural language processing to evaluate non-verbal cues, such as facial expressions and posture, and verbal responses to simulated interview questions. The platform will conclude each session with a comprehensive report detailing areas for improvement, effectively bridging the gap between theoretical preparation and practical application.

\section{Criteria for Study’s Success}

The success of this study will be determined by the following criteria:

\begin{itemize} \item Accuracy: The AI should accurately analyze facial expressions, posture, and verbal responses during interviews. \item Usability: The application should be user-friendly, making it accessible to a wide range of users with minimal technical expertise. \item Effectiveness: Feedback provided by the application should enable users to enhance their interview skills meaningfully. \item Engagement: The platform should simulate a realistic interview experience to engage users and increase preparation quality. \item Reliability: The tool should deliver consistent results across different sessions and scenarios. \end{itemize}