\chapter{Review of Literature} \label{ch
}

The literature review explores recent advancements in AI-powered interview platforms, pose and posture recognition, and real-time feedback mechanisms. This section provides a critical overview of state-of-the-art technologies in these areas, examining their methodologies, findings, and limitations. By synthesizing this information, this review identifies gaps in existing research and highlights the potential for further development in AI-driven interview preparation tools.

\section{Introduction} The purpose of this literature review is to establish a foundation for developing an AI-powered interview practice application that enhances interview readiness through real-time feedback on expressions, posture, and response quality. This review focuses on recent studies (within the last five years) that have explored interview simulations, pose detection technologies, and interactive feedback systems, aiming to provide insights into their contributions and limitations.

\section{Body} Each source is summarized and critically evaluated based on its research design, methodologies, and findings. The following section presents key studies relevant to this project, focusing on their approaches, contributions to interview preparation, and any identified gaps in technology or application.

\subsection{Tabular Analysis of Literature} The table below provides an organized summary of the reviewed literature, highlighting essential aspects such as methodologies, results, and limitations.

\begin{table}[h]
\centering
\begin{tabular}{|c|c|p{3.5cm}|p{3.5cm}|p{2.5cm}|p{3.5cm}|}
\hline
\textbf{Ref. No} & \textbf{Year} & \textbf{Basic Idea} & \textbf{Methodologies} & \textbf{Results} & \textbf{Limitations} \\ \hline
\cite{test} & 2024 & Mock interview platform with feedback & AI-powered interviews, personalized feedback & Improved interview skills & Not real-time monitoring, no posture correction \\ \hline
[2] & 2024 & Pose recognition technology (Viso.ai) & Deep learning & Accurate pose estimation for various applications & Detects only laying down, sitting, standing; no angle detection \\ \hline
[3] & 2024 & Interview warm-up platform & AI-powered practice questions, answer analysis & Increased confidence, interview preparedness & Only transcribes answers \\ \hline
[4] & 2023 & Hybrid posture detection framework & Machine learning, deep neural networks & Improved posture detection accuracy & High computational needs, data requirements, overfitting risk, real-time delays \\ \hline
[5] & 2022 & EZInterviewer, AI-driven tool for mock interviews & Natural language processing, machine learning & Personalized feedback, real-world simulation & Limited question variety, potential bias; no posture, facial detection \\ \hline
[6] & 2017 & Face detection, posture recognition for tracking systems & Machine learning (CNNs, keypoint detection) & Enhanced accuracy, efficiency in real-time face detection & Sensitive to lighting, occlusion issues, high computational demands, privacy concerns \\ \hline
\end{tabular}
\caption{Tabular Analysis of Reviewed Literature}
\label{tab:lit-review}
\end{table}


\section{Conclusion} The review of literature reveals that while there has been considerable progress in AI-powered interview and posture detection platforms, existing solutions often lack real-time monitoring, comprehensive posture correction, or a holistic assessment of interview skills. Notably, there is a gap in applications that combine posture analysis, facial expression tracking, and detailed response feedback in a single platform. These findings underscore the need for an integrated system capable of providing job seekers with a thorough, AI-driven assessment of their interview readiness. This project seeks to address these gaps by developing a comprehensive, real-time interview practice tool, informed by the insights gathered from existing research.


