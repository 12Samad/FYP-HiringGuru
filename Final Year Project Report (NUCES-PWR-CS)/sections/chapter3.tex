\chapter{Project Vision} \label{ch
}

This chapter provides an overview of the project's purpose, objectives, and anticipated impact. It discusses the problem the project aims to address, the business opportunities associated with it, and the project's scope, constraints, and stakeholders.

\section{Problem Statement} The current job market is highly competitive, with job seekers needing advanced skills and confidence to excel in interviews. Traditional interview preparation methods often lack personalized feedback on key elements like posture, facial expressions, and response quality. This project addresses the need for an interactive, AI-driven tool that provides real-time feedback on these aspects, empowering candidates to enhance their interview skills and improve their chances of success.

\section{Business Opportunity} There is a growing demand for professional development tools that leverage AI to enhance job seekers' readiness. This project presents an opportunity to capitalize on this trend by developing a platform that can be marketed to career development firms, educational institutions, and individuals seeking to improve their interview skills. The application's unique focus on real-time feedback for posture, expressions, and verbal responses distinguishes it from other interview preparation tools, creating a competitive advantage in the market.

\section{Objectives} The primary objectives of this project are as follows: \begin{itemize} \item Develop an AI-powered platform that enables users to practice interviews in a realistic, interactive environment. \item Provide real-time analysis of users' facial expressions, posture, and verbal responses. \item Generate comprehensive feedback reports to guide users in improving their interview skills. \item Ensure that the platform is user-friendly, accessible, and capable of delivering accurate, reliable feedback. \end{itemize}

\section{Project Scope} The project scope includes designing, developing, and deploying an AI-based interview practice platform with the following core features: \begin{itemize} \item Integration of webcam-based tracking to analyze facial expressions and body posture. \item An interactive interview simulation that asks users relevant questions based on their chosen field. \item Real-time feedback on user performance, focusing on non-verbal cues and response quality. \item A final report that highlights strengths, areas for improvement, and specific suggestions for enhancement. \end{itemize}

Features outside the current scope include advanced language processing beyond basic answer analysis, integration with external job platforms, and the ability to handle group interviews.

\section{Constraints} The project faces several constraints, including: \begin{itemize} \item Technical Constraints: Limitations in AI accuracy for facial and posture recognition in varied lighting or occluded environments. \item Computational Requirements: The need for high computational power for real-time processing, which may impact performance on lower-end devices. \item Privacy Concerns: Ensuring user data security and privacy during the collection, analysis, and storage of video data. \item Resource Limitations: Limited access to diverse datasets for training the AI models, which could impact the robustness of the feedback. \end{itemize}

\section{Stakeholders Description} The project involves various stakeholders, each with specific interests and goals associated with the application.

\subsection{Stakeholders Summary} The key stakeholders include: \begin{itemize} \item End Users: Job seekers who want to improve their interview skills. \item Educational Institutions: Universities and career development centers that can use the platform as a training tool. \item Career Development Firms: Organizations specializing in helping clients prepare for interviews. \item Development Team: Responsible for designing, developing, and maintaining the application. \end{itemize}

\subsection{Key High-Level Goals and Problems of Stakeholders} \begin{itemize} \item End Users: Seek a reliable, easy-to-use tool that offers actionable feedback for improving interview performance. They may face challenges in finding personalized guidance in conventional interview preparation resources. \item Educational Institutions: Aim to provide students with advanced resources to prepare for the job market, focusing on accessibility and integration with existing curricula. \item Career Development Firms: Interested in providing clients with the latest AI tools to gain an edge in the job market, but may have concerns about data privacy and accuracy of feedback. \item Development Team: Aims to build a robust, scalable, and user-friendly platform, with potential challenges related to technical feasibility, data handling, and resource management. \end{itemize}

